\section{Energy Shaping}
\showtoc

\subsection{Energy Shaping with Control Lyapunov Functions}

\begin{frame}[t]
  \frametitle{Motivation}
  \begin{block}{Main Question}
    Can we use an understanding of energy exchange to improve robustness  of
    periodic orbits in hybrid mechanical systems?
  \end{block}

  \begin{block}{Observations}
    \begin{itemize}
    \item Numerous control design schemes exist for stabilizing hybrid mechanical
      systems to periodic orbits.
    \item Some controllers produce good behavior locally but lack robustness.
    \item Periodic orbits have associated energy functions with level sets which
      are invariant under the orbits.
    \end{itemize}
  \end{block}
\end{frame}

\begin{frame}[t]
  \frametitle{Overview}
  \begin{block}{Main Idea}
    Add robustness to a periodic behavior by imposing convergence on a conserved
    energy function, $\Ec : \X \to \R$, to a level set which is known to be
    invariant under the system dynamics.
  \end{block}
  
  \begin{block}{Control Objective}
    Choose control input $\mu\arx$ such that $\| \mu\arx \|$ is minimized and
    $\Ec\arxt \to \Eref$ as $t \to \infty$.
  \end{block}

  \begin{block}{Exponential Convergence}
    To achieve exponential stabilization, $\Ec\arxt$ should satisfy\vspace{-.4em}
    \begin{align*}
      \Ec\arxt \leq \Ec\arxzero e^{-\beta t} \mbox{ for } t \geq 0, \beta > 0.
    \end{align*}
  \end{block}
\end{frame}

\begin{frame}[t]
  \frametitle{Conserved Energy Functions}
  \begin{block}{Conservative Systems}
    A conservative system with state space $\x = \argsqdq$ is modeled as
    \begin{align*}
      \HCSbar = \left\{
      \begin{array}{l l}
        \left.\begin{array}{r c l}
          \hspace{.58em}\dx &=& \xf\argsqdq + \xg\argsq \, \uu
        \end{array}\right\}  & \mbox{if } \argsqdq \in \D \setminus \S,\\
        \left. \begin{array}{r c l}
          \qp &=& \Deltaq\argsqm\\
          \dqp &=& \Deltadq\argsqdqm
        \end{array} \right\} & \mbox{if } \argsqdq \in \S.
      \end{array}\right.
    \end{align*}
    Total energy is conserved through the continuous dynamics, i.e.,
    \begin{align*}
      \Ec\argsqdq := \Tnrg\argsqdq + \Unrg\argsq.
    \end{align*}
  \end{block}
\end{frame}

\begin{frame}[t]
  \frametitle{Rapidly Exponentially Stabilizing CLFs}
  A \blue{rapidly exponentially stabilizing control Lyapunov function (\RESCLF)}
  $\V : \X \to \Rnn$ satisfying
  \begin{align*}
    &\cc_{1} \nx^{2} \leq \V\arx \leq \frac{\cc_{2}}{\resclfparam^{2}} \nx^{2},\\
    &\inf_{\uu \in \U} \Lie{\xf}\V\arx + \Lie{\xg}\V\arx \, \uu +
    \frac{\cc_{3}}{\resclfparam} \V\arx \leq 0
  \end{align*}
  for $\cc_{1}, \cc_{2}, \cc_{3} > 0$ guarantees exponential convergence of $\V\arx$, i.e.,
  \begin{align*}
    \V\arxt \leq \V(\xz) e^{-\frac{\cc_{3}}{\resclfparam} t} \mbox{ for } t \geq 0.
  \end{align*}
\end{frame}

\begin{frame}[t]
  \frametitle{Energy Shaping}
  Consider a conserved energy function $\Ec\arx$ on a hybrid control system
  \HCSbar which has a periodic orbit \orbit and define a Lyapunov candidate:
  \begin{align*}
    V\arx = \frac{1}{2} \left(\Ec\arx - \Eref\right)^{2},
  \end{align*}
  with $\Eref$ the constant energy level of the system on the orbit
  \orbit. For a \RESCLF, we seek a feedback control law, $\mu\arx$, such that
  \begin{align*}
    \Lie{\xf} V\arx + \Lie{\xg} \V\arx \, \mu\arx + \frac{\cc_{3}}{\resclfparam} \V\arx &\leq 0.
  \end{align*}
\end{frame}

\begin{frame}[t]
  \frametitle{Quadratic Program Formulation}
  The linear form of the \RESCLF condition suggests
  \begin{align*}
    \mueps\arx = \argmin_{\uu \in \R^{n}}  \, & \uu^T \uu,\\
    \mbox{s.t. } & \Aclf\arx \, \uu \leq \bclf\arx
  \end{align*}
  which encodes the dynamics of the system. This controller imposes exponential
  stabilization of the energy, i.e.,
  \begin{align*}
    \nxt \leq \frac{1}{\resclfparam}
    \sqrt{\frac{\cc_{2}}{\cc_{1}}} \nxz e^{-\frac{c_{3}}{2\resclfparam} t}.
  \end{align*}
\end{frame}

\begin{frame}[t]
  \frametitle{Equivalence of Hybrid Periodic Orbits}
  \begin{lemma}[Equivalence of Orbits]
    Applying the energy shaping controller to the hybrid control system $\HCSbar$
    results in the closed-loop hybrid system $\HSepsbar$ that demonstrates a
    periodic orbit which is identical to the unshaped system $\HSbar$.
  \end{lemma}
  \begin{block}{Proof}
    The energy of states $\xo \in \orbit$ is a constant,
    $\E(\xo) \equiv \Eref$. By the choice of Lyapunov function,
    $\V(\xo) \equiv 0$. In addition, $\dV(\xo) \equiv 0$ since
    the system is conservative. And since the solution to the optimization
    problem has cost $\uu^{T}(\xo) \uu(\xo) \equiv 0$, the
    control effort is also zero. Hence the orbits are equivalent.
  \end{block}
\end{frame}

\begin{frame}[t]
  \frametitle{Boundedness of \TtI{} Function}
  The difference in solutions between the shaped and unshaped systems can be
  bounded by examining the difference in solutions:
  \begin{lemma}[Boundedness of \TtI{}]
    For the hybrid systems $\HSbar$ and $\HSepsbar$,
    \begin{eqnarray}
      \nonumber
      &| \TIe(\Deltaxz) - \TI(\Deltaxz) | \leq \ATIe \nzdxzst,
    \end{eqnarray}
    where $\limeps \ATIe = 0$.
  \end{lemma}
\end{frame}


\begin{frame}[t]
  \frametitle{Main Theorem}
  \begin{block}{Theorem [Energy Shaping]}
    Given an exponentially-stable cycle in a hybrid system, application
    of the energy shaping controller results in the closed-loop hybrid system
    \begin{align*}
      \HS_{\resclfparam} = \left\{
        \begin{array}{r c l l}
          \dx &=& \xf\arx + \xg\arx \, \mueps\arx, & \x \in \D \setminus \S,\\
          \xp &=& \ResetMap(\xm), & \x \in \S,
        \end{array}\right.
    \end{align*}
    which is exponentially stable about the hybrid periodic orbit \orbit for
    large enough \resclfparam.
  \end{block}
\end{frame}

\begin{frame}[t]
  \frametitle{Overview of Proof}
  \begin{block}{Sketch of Proof [Energy Shaping]}
    \begin{enumerate}
    \item Transform the coordinates into a more intuitive form.
    \item Define a discrete Lyapunov candidate function valid on the \Poincare map.
    \item Show the conditions for stability through bounding arguments.
    \end{enumerate}
  \end{block}
\end{frame}

\begin{frame}
  \frametitle{Example: Compass Gait as a Shaped System}
  \only<1>{
    \begin{columns}
      \column{1.5in}
      Dynamic Model:
      \begin{align*}
        \M\argsq \ddq + \CG\argsqdq = \B\argsq \uu
      \end{align*}
      Control Law:
      \begin{align*}
        \vv\argsq &= \G\argsq - \G(\Psi\argsq)
      \end{align*}

      \column{1.5in}
      \begin{figure}
        \centering
        \def\svgwidth{1.0\columnwidth}
        \input{cg2d-2link-model.eps_latex}
        \vspace{-2em}
        \caption{Compass-gait biped with Controlled Symmetries.}
      \end{figure}
    \end{columns}
  }
  \only<2>{
    \begin{figure}
      \centering
      \includegraphics[width=1.0\columnwidth]{energy_cg2d_slope_model}
      \caption{Energy of the shaped system is conserved.}
    \end{figure}    
  }
  %  \only<3>{
  %    \includemedia[
  %      width=1.0\columnwidth,
  %      height=0.5625\columnwidth,
  %      addresource=amber2d.mp4,
  %      activate=pageopen,
  %      flashvars={source=amber2d.mp4&loop=true&autoPlay=true}
  %    ]{}{}%VPlayer9.swf}
  %  }

  \only<3>{
    Energy shaping can be achieved using:
    \begin{align*}
      \argmin_{u \in \Rn}  \, &\uu^{T} \uu\\
      \mbox{s.t. } & \Aclf\argsqdq \uu \leq \bclf\argsqdq
    \end{align*}
    where
    \begin{align*}
      \Aclf = 2 \eta \Lie{\xg} \eta, &&
      \bclf = -\resclfparam \eta^{2} - 2\eta \Lie{\xf} \eta,
    \end{align*}
    with
    \begin{align*}
      \eta = \Tnrg\argsqdq + \Unrg(\Psi\argsq) - \Eref.
    \end{align*}
  }

  \only<4>{
    \begin{figure}
      \centering
      \includegraphics[width=1.0\columnwidth]{es_comparison_2link_conservative}
      \caption{Demonstration of energy shaping on 2-link biped.}
    \end{figure}
  }

  \only<5>{
    \begin{figure}
      \centering
      \includemedia[
        %width=1.0\columnwidth,
        %height=0.5625\columnwidth,
        width=1.0\columnwidth,
        height=0.5\columnwidth,
        addresource=cg2d_es.mp4,
        activate=pageopen,
        flashvars={source=cg2d_es.mp4&loop=true&autoPlay=true}
      ]{}{VPlayer9.swf}
      \caption{Energy shaping to stabilize a gait from a distant initial condition.}
    \end{figure}
  }

  \only<6> {
    \begin{figure}
      \centering
      \subfloat[Nominal system.]{\includegraphics[width=.48\textwidth]{pp_eps_0}}
      \subfloat[Shaped system with $\resclfparam =
      \frac{1}{100}$.]{\includegraphics[width=.48\textwidth]{pp_eps_1_100}}
      \caption{Comparison of unshaped and shaped systems.}
    \end{figure}
  }

  \only<7> {
    \begin{figure}
      \centering
      \includegraphics[height=.65\textheight]{poincare_cloud}
      \caption{A three-dimensional representation of the domain of attraction
        on the guard, comparing the unshaped and shaped system.}
    \end{figure}
  }
\end{frame}
